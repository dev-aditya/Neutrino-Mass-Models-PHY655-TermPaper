\section{Implication of Neutrino Mass,  Limitations and Open Questions}

Neutrino masses are known to be nonzero. They are also tiny when compared with all other mass scales in the SM. Non-zero neutrino masses are bonafide evidence for \textit{physics beyond the SM}. New ingredients must be added to the our current understanding of particle physics in order to explain the phenomena revealed by neutrino oscillation experiments. But there are a lot of questions yet to be answered and understood and knowing the correct answer to these questions could lead to paradigm-shifting developments in physics and astrophysics.  Example: The unitarity of the transformation connecting the mass eigenstates to the electroweak eigenstates is not firmly established. There are tantalizing hints, but no firm evidence, of the existence of sterile neutrinos that do not couple to the vector bosons of the Standard Model, but nevertheless mix with the neutrinos that do. We do not know the transformation properties of the neutrinos under particle-antiparticle conjugation (i.e. whether the neutrinos are Majorana or Dirac fermions). We have just started exploring the full potential of the neutrinos in astrophysics and cosmology. We have just glimpsed into the whole new world of neutrinos, a new domain of physics which stands waiting to be explored!


Following what we discusses in earlier section, the following possibilities arise. Either: 
\begin{itemize}
    \item the neutrino is a Dirac particle, and hence there must be a right-handed chiral neutrino state, which does not interact with matter. This is a so-called \textit{sterile neutrino}.
    \item the neutrino is a Majorana particle. In this case, the neutrino and the antineutrino are identical. The mass term directly couples the left-handed neutrino with the right-handed antineutrino, which implies that the neutrino must have mass. Further, such an interaction implies that lepton number is violated by 2.
    \item If the neutrino is Majorana, then it is possible to explain the very low value of the neutrino mass compared to the quark and charged lepton masses at the expense of the introduction of another very heavy Majorana neutrino though the see-saw mechanism. \textsc{CP} violation in the decays of this heavy neutrino in the early universe could have created the baryon asymmetry we see today.
\end{itemize}



The possibility of the existence of heavy neutrinos, as discussed in  Section ~\ref{subsec:seesaw_mechanism}, also have given rise to another intriguing idea called \textit{leptogenesis} which is intrinsically linked to the question of the matter-antimatter asymmetry. The idea is that these very heavy neutrinos, which are Majorana particles, decayed as the universe cooled into lighter left-handed neutrinos or right-handed antineutrinos, along with Higgs bosons, which themselves decayed to quarks. If the probability of one of these heavy neutrinos to decay to a left-handed neutrino was slightly different than the probability to decay to a right-handed anti-neutrino, then there would be a greater probability to create quarks than anti-quarks and the universe would be matter dominated. More formally, it is thought that the quantum number B - L, where B is the baryon number of the universe and L is the lepton number, must be conserved. If there was some violation of L in the decays of the heavy Majorana neutrinos, this would manifest as a violation in B, and hence the missing anti-matter problem could actually arise from CP violation in the neutrinos. Although there is no direct connection between CP violation in the heavy neutrinos and CP violation in the light neutrinos, this idea still motivates the current attempt to measure CP violation in the light neutrinos at today’s long baseline neutrino experiments.
