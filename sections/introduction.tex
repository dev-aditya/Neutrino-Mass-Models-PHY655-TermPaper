
\section{\label{sec:intro}Introduction}


Neutrinos are elementary particles that are electrically neutral and interact only weakly with matter. They were first postulated by Wolfgang Pauli in 1930 to explain the apparent violation of energy conservation in certain types of radioactive decay. Since then, neutrinos have been detected in a variety of astrophysical and laboratory settings, including nuclear reactors, particle accelerators, and the cosmic microwave background.

For many years, it was assumed that neutrinos were massless, due to their weak interactions and the absence of any direct evidence for their mass. However, in the late 1990s, a series of experiments provided the first evidence that neutrinos have non-zero mass. This discovery was a major breakthrough in particle physics and has led to a renewed interest in understanding the properties of neutrinos.

There are several models of neutrino mass that have been proposed over the years, each with its own strengths and weaknesses. These models include the seesaw mechanism, which postulates the existence of heavy right-handed neutrinos, and the radiative seesaw mechanism, which explains the smallness of neutrino mass through radiative corrections. Other models include the inverse seesaw mechanism, the low-scale leptogenesis scenario, and the Majorana model.

Neutrinos have been demonstrated to be priceless probes when it comes to understanding the weak interactions and revealing the internal structure of nucleons and nuclei and has important implications for particle physics and cosmology. Neutrinos are produced abundantly in nature, most notably in circumstances where the weak interactions play a decisive role. These include radioactive decay processes and the nuclear fusion processes that occur inside stars. The measurement of neutrinos produced in the core of the Sun helped establish that the Sun’s energy is a consequence of nuclear fusion dominated by the so-called \texttt{pp} chain.

\subsection{\label{subsec:history_of_nu}Discovering Neutrinos: History and Detection of Neutrino}

For two-body decays, the energy of each of the decay products in the rest frame of the decaying particle is a fixed quantity. Therefore, the observation of continuous electron energy spectra for beta decays of nuclei, where only the daughter nucleus and the emitted electron were observed, posed a puzzle. Either the conservation of energy and momentum was in peril, or something was missing. Pauli proposed (1932) that this missing something is a chargeless and massless particle, termed “neutrino”.

Introducting the hypothetical particle explained both the energy-momentum conservation and the shape of electron spectrum that was observed: 
\begin{eqnarray}
    \dfrac{d \Gamma}{d E_e} \propto p_e E_e\left(E_0 - E_e\right) \sqrt{\left(E_0 - E_e\right)^2 - m_\nu ^2}
\end{eqnarray}
where \(p_e\) and \(E_e\) are momentum and energy of electron and \(E_0 \equiv Q - m_e - m_\nu\) is the maximum energy of electron. 

In the limit when neutrino is massless, the slope of the ``Kurie Plot" is a constant:
\begin{eqnarray}
    \dfrac{d\Gamma / dt}{p_e E_e} \propto \left[(E_0 - E_e)\sqrt{(E_0 - E_e)^2 - m_\nu ^2}\right]
\end{eqnarray}
The beta decay experiments have put an upper limit on electron neutrino mass of \(2.2~\mathrm{eV}\). 

The first direct observation of neutrinos was by Reines and Cowan in 1956\cite{cowan1956detection}, wherein they directed a flux of \(\overline {\nu} _e\) (supposed to have come from a beta decay) into a water target. The reaction \(\overline {\nu} _e p \to n e^+\) produced positrons, which annihilated with electrons in the scintillation counters giving two $0.5 ~\mathrm{MeV}$ photons. The neutrons were absorbed by $\mathrm{CdCl_2}$ dissolved in water, which emitted photons within a few $\mu s$. The coincidence of these two kinds of photons confirmed the above reaction, and hence the presence of \(\overline {\nu} _e\). 

The discovery of \({\nu} _\mu\) took place in 1962 at the Brookhaven National Laboratory \cite{brookhaven} through the decays of pions. Iron from the USS battleship Missouri was used as the target. It was observed that the interactions of these neutrinos with the nuclei (\(\nu_\mu /\overline {\nu}_\mu +N \to \mu _ - /\mu _+ + N' \)) produced only muons, but no electrons. This showed that the ``muon" neutrinos produced in pion decay were distinct from the electron neutrinos produced in beta decays. This was an indication of ``lepton flavour conservation" (which we now know does not hold true in general). The direct observation of \(\nu _\tau\) took place only very recently, in 2000 at the DONUT experiment \cite{KODAMA2001218} at CERN. 